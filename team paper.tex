\documentclass[12pt,fleqn]{article}

\usepackage{amsmath,amssymb,amsthm,enumerate,color,graphics,epsfig,url}

\usepackage{cmap}

\usepackage[pdftex,
colorlinks,%
linkcolor=blue,citecolor=red,urlcolor=blue,
hyperindex,%
plainpages=false,%
bookmarksopen,%
bookmarksnumbered,%
unicode]{hyperref}
\usepackage [dvipsnames] {xcolor}
\usepackage{pgf,tikz,pgfplots,tikz-3dplot}
\usetikzlibrary{calendar,folding}
\usetikzlibrary{arrows,patterns,decorations.pathmorphing,backgrounds,positioning,fit,petri}
\usetikzlibrary{calc,3d,intersections,shapes}
%\pgfplotsset{compat=1.11}
\usepgfplotslibrary{polar}

\usepackage[utf8]{inputenc}
\usepackage[ukrainian]{babel}

\setlength{\textwidth}{160.0mm}
\setlength{\textheight}{240.0mm}
\setlength{\oddsidemargin}{0mm}
\setlength{\evensidemargin}{0mm}
\setlength{\topmargin}{-18mm}
\setlength{\parindent}{5.0mm}

\newtheorem{theorem}{Теорема}[section]
\newtheorem*{theoremN}{Теорема}
\newtheorem{proposition}{Твердження}[section]
\newtheorem{statement}{Твердження}[section]
\newtheorem{lemma}{Лема}[section]
\newtheorem{corollary}{Наслідок}[section]

{\theoremstyle{definition}
\newtheorem{definition}{Означення}[section]
\newtheorem{example}{Приклад}[section]
\newtheorem{problem}{Задача}[section]
\newtheorem*{problem*}{Задача}
\newtheorem{question}{Питання}[section]

}

\usepackage[labelsep=period]{caption}
\numberwithin{figure}{section}
\numberwithin{equation}{section}

\begin{document}

\vspace{50mm}

\begin{center}
\Large\bf
Конспект\\[50mm]
{\Huge}
\end{center}

\vspace{50mm}

\newpage

\tableofcontents

%\listoffigures

\newpage


\section{189}\label{189}\allowdisplaybreaks

тому час, який минув між $(i-1)$-м та $i$-м викликами дорівнює $X_{i}^{*}$. Нехай
$$N_{t}^{*}=\max\{k: T_{k}^{*}\leq t\}$$
це кількість дзвінків, які надійшли за час $t$. Тоді процес $N^{*}=(N_{t}^{*}:t\geq 0)$ є процесом Пуассона з параметром $\lambda$, тому зі сторони Білла, новий телефон працює подібно (за статистикою), що й минула модель.

\begin{example}
    Припустимо, що прибуття автобусів від центра міста до зупинки --- це випадкова величина розподілена за законом Пуассона. Знаючи це, Девід  очікує експоненціально розподілений період часу, перш ніж його забере автобус.  Якщо він прибуває на автобусну зупинку, і Доріс  каже йому, що вона чекає вже 50 хвилин, то ця інформація не несе ніякої цінності для нього, оскільки експоненціальний розподіл має властивість відсутності післядії. Аналогічно, якщо він прибуває саме вчасно, щоб побачити автобус, який відправляється, то йому не варто хвилюватися, що він чекатиме довше, ніж зазвичай. Ці властивості є характеристиками процесу Пуассона.
\end{example}

\begin{problem}
     Нехай $M$ та $N$ --- незалежні процеси Пуассона, $M$ має параметр $\lambda$, а $N$ --- параметр $\mu$. Використайте результат задачі 6.9.4, щоб показати, що процес $M + N = (M_t + N_t : t \geq 0)$ є процесом Пуассона з параметром $\lambda + \mu$. Порівняйте цей метод із методом вправи 11.5.
\end{problem}

\begin{problem}
  Якщо $T_i$ є часом $i$-го надходження в Пуассоніський процес $N$, то довести, що $N_t < k$ тоді і тільки тоді, коли $T_k > t$. Використовуйте теорему 11.23 і центральну граничну теорему. Теорема 8.25, щоб вивести, що $t \rightarrow \infty$,
  $$ \mathbb{P}(\frac{N_t-\lambda t}{\sqrt{\lambda t}}\leq u)\rightarrow \int_{-\infty}^{u}\frac{1}{\sqrt{2\pi}}e^{-\frac{1}{2}v^2}\, \mathrm{d}v \quad \text{для $u \in \mathbb{R}.$} $$
  !
  Порівняйте це з задачею 11.20.
\end{problem}

\begin{problem}
  Дзвінки надходять на телефонну станцію за розподілом Пуассонівсього процесу з параметром $\lambda$, але оператор часто відволікається і відповідає лише на кожен другий дзвінок. Чому дорівнює загальний інтервал часу, що проходить між послідовними викликами, які були опрацьовані оператором?
\end{problem}

\subsection{Природний приріст населення та прості процеси народження}\label{11.4}
Ідеї останніх розділів багато застосувань, однією з яких є модель природного росту, яка відбувається впродовж безперервного часу. Мається на увазі проста модель явища, як прогресивний поділ клітин амеби, тому сформулюємо процес за допомогою цих термінів. Гіпотетичний вид амеби розмножується наступним чином.Скажімо, у момент часу $t=0$ є початкова кількість амеб, це число $I$, які знаходяться у великому ставку. З плином часу кількість цих амеб збільшується завдяки прогресивному процесу розподілу клітин. Коли амеба ділиться, одна батьківська амеба замінюється \emph{точно} двома ідентичними копіями самої \emph{себе}. Кількість амеб у ставку зростає в продовж часу, але ми не можемо упевненістю сказати, якою буде кількість в майбутньому, оскільки поділ клітин відбувається у \emph{випадковий} час період часу (як у попередній моделі, телефонні дзвінки надходять випадково). Зробимо наступні два припущення щодо цього процесу.\footnote{На практиці амеби та бактерії розмножуються з параметром, який залежить від їхнього середовища. Хоча існують значні відмінності між життєвими циклами клітин в середовищі, як правило їм не вистачає ступенів гомогенності та незалежності, які наведені в цьому прикладі.}.

\textbf{Швидкість поділу}: в момент часу $t$ кожна амеба, яка присутня в ставку має ймовірність поділу впродовж короткого інтервалу часу $(t, t + h)$. Існує константа $\lambda (>0)$, яка називається \emph{коефіцієнтом народжуваності}, така що ймовірність того, що будь-яка така амеба

\begin{enumerate}
  \item ділиться, \emph{один раз} у інтервалі часу $(t, t+ h]$ дорівнює $\lambda h + o(h)$,
  \item \emph{не ділиться}, в інтервалі часу $(t, t+ h]$, дорівнює $1 - \lambda h + o(h)$.
\end{enumerate}

\textbf{Незалежність} у момент часу $t$ для кожної амеби усі майбутні поділи відбуваються незалежно від кількості минулих поділів і від (минулої та майбутньої) діяльності усіх інших амеб, які відбулися в момент часу $t$.

Кожна амеба, яка присутня в момент часу $t$, має ймовірність $\lambda h + o(h)$ поділитися на дві амеби за час $t+h$, відповідно за час $t+h$ ймовірність породженості однієї амеби (самої) дорівнює $1 - \lambda h + o(h)$. Отже, за час $t+h$, ймовірність породженості більше двох амеб дорівнює $o(h)$. Нехай $M_t$ --- кількість амеб, які присутні в момент часу $t$. З попередніх тверджень, можна дійти висновку, що $M_{t+h}$ залежить від кількості амеб $M_t$. Припустимо, що $M_t = k$. Тоді $M_{t+h}\geq k$, і
\begin{multline}
  \mathbb{P}(M_{t+h}=k\big| M_t=k) = \mathbb{P}(\text{поділ не відбувся}) =  \\
  [1-\lambda h + o(h)]^k \\
  1 - \lambda k h + o(h).
\end{multline}

Також
\begin{equation}
\mathbb{P}(M_{t+h}=k+1 \big | M_t = k) = \mathbb{P}(\text{тільки один поділ})\\
=\bigg(\begin{matrix} k \\ 1 \end{matrix}\bigg)[\lambda h + o(h)][1-\lambda h + o(h)]^{k-1}\\
\lambda k h + o(h),
\end{equation}

Оскільки, $k$ --- можливий вибір для поділу клітин. Нарешті
\begin{equation}
\mathbb{P}(M_{t+h}\geq k + 2 \big| M_t = k) = 1 - \mathbb{P}(M_{t+h} \text{є} k \text{або} k+1 \big| M_t=k)
\qquad{} = 1- [\lambda kh + o(h)] - [1-\lambda kh + o(h)]
\\= o(h).
\end{equation}

Отже, процес $M$ розвивається майже так само, як і процес Пуассона $N$, тобто якщо $M_t = k$, то $M_{t+h} = k$, або відповідно $M_{t+h} = k+1$ з ймовірністю $1-o(h)$. Велика різниця між $M$ та $N$ наведена у порівнянні (11.34) і (11.1) відповідно. Це означає, що параметр, з яким $M$ зростає, пропорційний самому $M$, в той час як процес Пуассона зростає з постійним параметром. Процес $M$ називається \emph{простим (лінійним) процесом народження} або \emph{чистим процесом народження}. Ми доводимо це за допомогою тих самих методів та тверджень, які використовували для процесу Пуассона.

\begin{theorem}
  Якщо $M_0$ і $t>0$, тоді
  \begin{equation}\label{11.37}
    \mathbb{P}(M_t=k) = \bigg(\begin{matrix}
                          k-1 \\
                          I-1
                        \end{matrix}\bigg) e^{-I\lambda t}(1-e^{\lambda t})^{k-I}\quad \text{для} k=I, I+1, \dots
  \end{equation}
\end{theorem}

\begin{proof}
  Нехай,
  $$p_k(t)=\mathbb{P}(M_t=k)$$
  як і раніше. Встановимо диференціально-різницеві рівняння для $p_I(t), p_{I+1}(t), \dots$ так само як ми знайшли (11.11) і (11.12). Отже, з теореми про розбиття отримаємо, що для $h>0$,
  \[\mathbb{P}(M_{t+h}=k)=\sum_{i}\mathbb{P}(M_P{t+h}=k\big|M_i=i)\mathbb{P}(M_i=i)
     \\ =\left[1-\lambda kh +o(h)\right]\mathbb{P}(M_t=k)+\left[\lambda(k-1)h+o(h)\right]\mathbb{P}(M_t=k-1)+o(h)\]
  використавши (11.33)-(11.35), отримаємо
  \[p_k(t+h)-p_k(t)=\lambda(k-1)hp_{k-1}(t)-\lambda khp_k(t) + o(h). \]
  Поділимо це рівняння на $h$ і візьмемо межу від $h\downarrow0$, щоб отримати 
  \[p_{k}'(t)=\lambda(k-1)p_{k-1}(t)-\lambda kp_k(t) \quad \text{для } k=I, I+1, \dots.\]
  Рівняння для $p_{I}'(t)$ містить $p_{I-1}(t)$, також зауважимо, що $p_{I-1}(t)=0$ для всіх $t$. Тепер рекурсивно розв'яжемо (11.38) з дотриманням граничної умови 
  \begin{equation}\label{11.39}
  p_{k}(0)=\begin{matrix}
             1 \quad \text{якщо } k=I, \\
             0 \quad \text{якщо } k\neq I.
           \end{matrix}
  \end{equation}
  Тобто спочатку знайдемо $p_{I}(t)$, потім $p_{I+1}(t), \dots,$ і за допомогою математичної індукції доводимо формулу (11.37).
  
  Також можемо використати метод генерування функцій. Якщо помножити (11.38) на $s^k$ і підсумуємо по $k$, то отримаємо диференціальне рівняння з частинними похідними
  \begin{equation}\label{11.40}
    \frac{\mathrm{d}G}{\mathrm{d}t}=\lambda s(s-1)\frac{\mathrm{d}G}{\mathrm{d}s}
  \end{equation}
  де $G=G(s, t)$ --- генератриса
  $$G(s,t)=\sum_{k=I}^{\infty}p_k(t)s^k.$$
  Розв'язати це диференціальне рівняння з дотриманням граничної умови $G(s,0)=s^I$ не є складним, але ми не потребує таких навичок від читача.
\end{proof}

Середнє значення та дисперсію $M_t$ можна обчислити безпосередньо за допомогою (11.37) звичайним способом. Ці обчислення дещо складніші, оскільки $M_t$ містить від'ємний біноміальний розподіл, і простіше виконати наступну дію. Запишемо
$$\mu(t)=\mathbb(E)(M_t)=\sum_{k=I}^{\infty}kp_k(t),$$
Отримали, диференціюючи через підсумовування знову, що
\begin{equation}\label{11.41}
  \mu'(t)=\sum_{k=I}^{\infty}kp_k'(t)=\sum_{k=I}^{\infty}k\left[\lambda(k-1)p_{k-1}(t)-\lambda k p_k(t) \right]
\end{equation}
з (11.38). Сумуємо коефіцієнти, щоб отримати $p_k(t)$, щоб отримати
\begin{equation}\label{11.42}
  \mu'(t)=\lambda\sum_{k=I}^{\infty}\left[(k+1)kp_{k}(t)-k^2p_k(t) \right]
\end{equation}
\begin{equation}\label{11.43}
  = \lambda \sum_{k=I}^{\infty} kp_k(t)=\lambda\mu(t),
\end{equation}
яке є диференціальним рівнянням $\mu$ із граничною умовою
$$\mu(0)=\mathbb{E}(M_0)=I.$$
Дане диференціальне рівняння має розв'зок
$$\mu(t)=Ie^{\lambda t},$$
показуючи, що (в середньому) амеби розмножуються з експоненціальною швидкістю (тоді як процес Пуассона в середньому зростає лінійно, запам'ятайте (11.8)). Такий самий тип аргументу можна використовувати для обчислення $\mathbb{E}(M_{t}^{2})$. Це більш складний процес і призводить до виразу з дисперсією $M_t$,
\begin{equation}\label{11.44}
  \text{var}(M_t)=Ie^{2\lambda t}(1-e^{-\lambda t}).
\end{equation}

Альтернативний метод обчислення середнього значення та дисперсії $M_t$ здійснюється за допомогою диференціального рівняння (11.40) для функції, що створює йморівність $G(s,t)$ від $M_t$. Пам'ятайте, що
$$G(1,t)=1, \quad \frac{\mathrm{d}G}{\mathrm{d}s}\bigg|_{s=1}=\mu(t). $$
Ми диференціюємо все рівняння (11.40) відносно $s$ і підставляємо $s=1$,  щоб отримати
$$\frac{\mathrm{d}^2G}{\mathrm{d}s\mathrm{d}t}\bigg|_{s=1}=\lambda\frac{\mathrm{d}G}{\mathrm{d}s}\bigg|_{s=1}.$$
Припустимо, що ми можемо змінити порядок диференціювання в першому члені, тоді рівняння набуває вигляду 
$$\mu'(t)=\lambda\mu(t)$$
відповідно до (11.43). Дисперсію можна знайти подібним чином, за допомогою подвійного диференціювання.

\begin{example}\label{11.46}
  Довести, що в простому процесі народження, описаному вище, період часу, впродовж якого існує рівно $k(\geq I)$ особин, є випадковою змінною, яка має експоненціальний розподіл з параметром $\lambda k$.
\end{example}

\begin{example}\label{11.47}
  Зробити висновок із результату, який отримали у вправі (11.46), що час $T_{I,J}$, який необхідний процесу народження для зростання розміру $I$ до розміру $J (>I)$, має середнє значення та дисперсію, яка задана рівнянням
  $$\mathbb{E}(T_{I,J)}=\sum_{k=I}^{J-I}\frac{I}{\lambda k}, \quad \text{var}(T_{I,J})=\sum_{k=I}^{J-I}\frac{I}{(\lambda k)^2}$$
\end{example}

\end{document} 