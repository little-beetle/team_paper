\documentclass[12pt,fleqn]{article}

\usepackage{amsmath,amssymb,amsthm,enumerate,color,graphics,epsfig,url}

\usepackage{cmap}

\usepackage[pdftex,
colorlinks,%
linkcolor=blue,citecolor=red,urlcolor=blue,
hyperindex,%
plainpages=false,%
bookmarksopen,%
bookmarksnumbered,%
unicode]{hyperref}
\usepackage [dvipsnames] {xcolor}
\usepackage{pgf,tikz,pgfplots,tikz-3dplot}
\usetikzlibrary{calendar,folding}
\usetikzlibrary{arrows,patterns,decorations.pathmorphing,backgrounds,positioning,fit,petri}
\usetikzlibrary{calc,3d,intersections,shapes}
%\pgfplotsset{compat=1.11}
\usepgfplotslibrary{polar}

\usepackage[utf8]{inputenc}
\usepackage[ukrainian]{babel}

\setlength{\textwidth}{160.0mm}
\setlength{\textheight}{240.0mm}
\setlength{\oddsidemargin}{0mm}
\setlength{\evensidemargin}{0mm}
\setlength{\topmargin}{-18mm}
\setlength{\parindent}{5.0mm}

\newtheorem{theorem}{Теорема}[section]
\newtheorem*{theoremN}{Теорема}
\newtheorem{proposition}{Твердження}[section]
\newtheorem{statement}{Твердження}[section]
\newtheorem{lemma}{Лема}[section]
\newtheorem{corollary}{Наслідок}[section]

{\theoremstyle{definition}
\newtheorem{definition}{Означення}[section]
\newtheorem{example}{Приклад}[section]
\newtheorem{problem}{Задача}[section]
\newtheorem*{problem*}{Задача}
\newtheorem{question}{Питання}[section]

}

\usepackage[labelsep=period]{caption}
\numberwithin{figure}{section}
\numberwithin{equation}{section}

\begin{document}

\vspace{50mm}

\begin{center}
\Large\bf
Конспект\\[50mm]
{\Huge}
\end{center}

\vspace{50mm}

\newpage

\tableofcontents

%\listoffigures

\newpage


\section{189}\label{189}\allowdisplaybreaks

! $(i-1)$ ! $X_{i}^{*}$.
$$N_{t}^{*}=\max\{k: T_{k}^{*}\leq t\}$$
! $t$. ! $N^{*}=(N_{t}^{*}:t\geq 0)$ ! $\lambda$, !.

\begin{example}
    !
\end{example}

\begin{problem}
    ! $M$ ! $N$ !, $M$ ! $\lambda$ ! $N$ ! $\mu$. ! $M + N = (M_t + N_t : t \geq 0)$ ! $\lambda + \mu$. !
\end{problem}

\begin{problem}
  ! $T_i$ ! $N$, ! $N_t < k$ ! $T_k > t$. ! $t \rightarrow \infty$,
  $$ \mathbb{P}(\frac{N_t-\lambda t}{\sqrt{\lambda t}}\leq u)\rightarrow \int_{-\infty}^{u}\frac{1}{\sqrt{2\pi}}e^{-\frac{1}{2}v^2}\, \mathrm{d}v \quad \text{для $u \in \mathbb{R}.$} $$
  !
\end{problem}

\begin{problem}
  ! $\lambda$
\end{problem}

\subsection{!}\label{11.4}
! $t=0$ ! $I$

Division rate! $t$ ! $(t, t + h)$ ! $\lambda (>0)$

! $(t, t+ h]$ ! $\lambda h + o(h)$
! $(t, t+ h]$ ! $1 - \lambda h + o(h)$

Independence! $t$ ! $t$.

! $t$ ! $\lambda h + o(h)$ ! $t+h$, ! $1 - \lambda h + o(h)$ ! $t+h$, ! $o(h)$ ! $t+h$. ! $M_t$ ! $t$. ! $M_{t+h}$ ! $M_t$. ! $M_t = k$. ! $M_{t+h}\geq k$, !

\begin{equation}
\mathbb{P}(M_{t+h}=k\big| M_t=k) = \mathbb(P)\text{!}
\\=[1-\lambda h + o(h)]^k
\\1 - \lambda k h + o(h).
\end{equation}
!,

\begin{equation}
\mathbb{P}(M_{t+h}=k+1 \big | M_t = k) = \mathbb{P}(\text{!})\\
=\bigg(\begin{matrix} k \\ 1 \end{matrix}\bigg)[\lambda h + o(h)][1-\lambda h + o(h)]^{k-1}\\
\lambda k h + o(h),
\end{equation}

! $k$ !
\begin{equation}
\mathbb{P}(M_{t+h}\geq k + 2 \big| M_t = k) = 1 - \mathbb{P}(M_{t+h} is k or k+1 \big| M_t=k)
\\= 1- [\lambda kh + o(h)] - [1-\lambda kh + o(h)]
\\= o(h).
\end{equation}

! $M$ ! $N$ ! $M_t = k$ ! $M_{t+h} = k$ ! $M_{t+h} = k+1$ ! $1-o(h)$. ! $M$ ! $N$ ! $M$ ! $M$ ! $M$.

\begin{theorem}
  ! $M_0$ ! $t>0$ !
  \begin{equation}\label{11.37}
    \mathbb{P}(M_t=k) = \bigg(\begin{matrix}
                          k-1 \\
                          I-1
                        \end{matrix}\bigg) e^{-I\lambda t}(1-e^{\lambda t})^{k-I}\quad \text{for $k=I, I+1, \dots$}
  \end{equation}
\end{theorem}

\begin{proof}
  !
  $$p_k(t)=\mathbb{P}(M_t=k)$$
  ! $p_I(t),p_{I+1}(t), \dots$ ! $h>0$,
  \[\mathbb{P}(M_{t+h}=k)=\sum_{i}\mathbb{P}(M_P{t+h}=k\big|M_i=i)\mathbb{P}(M_i=i)
    \\=\left[1-\lambda kh +o(h)\right]\mathbb{P}(M_t=k)+\left[\lambda(k-1)h+o(h)\right]\mathbb{P}(M_t=k-1)+o(h)\]
  !
  \[p_k(t+h)-p_k(t)=\lambda(k-1)hp_{k-1}(t)-\lambda khp_k(t) + o(h). \]
  ! $h$ ! $h\downarrow0$
  \[p_{k}'(t)=\lambda(k-1)p_{k-1}(t)-\lambda kp_k(t) \quad \text{для } k=I, I+1, \dots\]
  ! $p_{I}'(t)$ ! $p_{I-1}(t)$ ! $p_{I-1}(t)=0$ ! $t$. !
  \begin{equation}\label{11.39}
  p_{k}(0)=\begin{matrix}
             1 \quad \text{якщо } k=I, \\
             0 \quad \text{якщо } k\neq I.
           \end{matrix}
  \end{equation}
  ! $p_{I}(t)$, ! $p_{I+1}(t), \dots,$ !.
  
  ! $s^k$ ! $k$ !
  \begin{equation}\label{11.40}
    \frac{\mathrm{d}G}{\mathrm{d}t}=\lambda s(s-1)\frac{\mathrm{d}G}{\mathrm{d}s}
  \end{equation}
  ! $G=G(s, t)$ !
  $$G(s,t)=\sum_{k=I}^{\infty}p_k(t)s^k.$$
  ! $G(s,0)=s^I$.
\end{proof}

! $M_t$ ! $M_t$ !

$$\mu(t)=\mathbb(E)(M_t)=\sum_{k=I}^{\infty}kp_k(t),$$
!
\begin{equation}\label{11.41}
  \mu'(t)=\sum_{k=I}^{\infty}kp_k'(t)=\sum_{k=I}^{\infty}k\left[\lambda(k-1)p_{k-1}(t)-\lambda k p_k(t) \right]
\end{equation}
!
\begin{equation}\label{11.42}
  \mu'(t)=\lambda\sum_{k=I}^{\infty}\left[(k+1)kp_{k}(t)-k^2p_k(t) \right]
\end{equation}
\begin{equation}\label{11.43}
  = \lambda \sum_{k=I}^{\infty} kp_k(t)=\lambda\mu(t),
\end{equation}
\end{document} 